%%%%%%%%%%%%%%%%%%%%%%%%%%%%%%%%%%%%%%%%%
% Twenty Seconds Resume/CV
% LaTeX Template — Version française
% Version 1.0 (14/7/16)
%
% Original author:
% Carmine Spagnuolo (cspagnuolo@unisa.it) with major modifications by
% Vel (vel@LaTeXTemplates.com) and Harsh (harsh.gadgil@gmail.com)
%
% License:
% The MIT License (see included LICENSE file)
%
%%%%%%%%%%%%%%%%%%%%%%%%%%%%%%%%%%%%%%%%%

%----------------------------------------------------------------------------------------
%	PACKAGES AND OTHER DOCUMENT CONFIGURATIONS
%----------------------------------------------------------------------------------------

\documentclass[letterpaper]{twentysecondcv} % a4paper for A4

% Command for printing skill overview bubbles
\newcommand\skills{
~
	\smartdiagram[bubble diagram]{
        \textbf{Cloud}\\\textbf{Architect},
        \textbf{SDLC}\\\textbf{DevOps},
        \textbf{~~~~~~~~Sécurité~~~~~~~~~},
        \textbf{~~~~~~Conteneurs~~~~~~}\\\textbf{~~(Docker/K8s)~~},
        \textbf{~~~~~Automatisation~~~~~}
    }
}

% Barres de compétences (format: {libellé / niveau} ; un "/" par élément)
\programming{{Python Shell Java C++ / 5}, {Sécurité Monitoring / 4}, {Docker Kubernetes SDLC / 5}}

% Formation
\education{
\textbf{Diplôme d'Ingénieur, Informatique} \\
EFREI - École d'Ingénieurs Généralistes \\
1999 - 2004 | Paris, France

\textbf{Baccalauréat} (Série scientifique, Mathématiques) \\
1999 | France
}

%----------------------------------------------------------------------------------------
%	 INFORMATIONS PERSONNELLES
%----------------------------------------------------------------------------------------

\cvname{Alban ANDRIEU}
\cvjobtitle{Architecte Cloud DevSecOps}

\cvlinkedin{/in/nabla}
\cvgithub{} % laisser vide si non utilisé
\cvnumberphone{+33 6 95 43 53 XX}
\cvsite{nabla.albandrieu.com}
\cvmail{alban.andrieu@free.fr}

%----------------------------------------------------------------------------------------

\begin{document}

\makeprofile % Afficher la barre latérale

%----------------------------------------------------------------------------------------
%	 EXPÉRIENCE
%----------------------------------------------------------------------------------------

\section{Expérience}

\begin{twenty}
	\twentyitem
		{2022 -}
		{Présent}
		{Architecte Cloud, Intégration (DevSecOps)}
		{\href{https://jusmundi.com}{Jus Mundi}}
		{Paris, France}
		{\begin{itemize}
		\item \textbf{Mission :} Transition IA et hébergement du site pour la plateforme droit international \& arbitrage (jusmundi.com)
		\item \textbf{Sécurité infrastructure :} Analyse de vulnérabilités, gestion des secrets, contrôles d'accès
		\item \textbf{Conteneurisation \& Kubernetes :} Images Docker, clusters K8s en production (réseau, stockage, supervision)
		\item \textbf{SDLC :} Standardisation des processus, indicateurs qualité, coordination inter-équipes avec Dev et Data Science
		\item \textbf{Stack :} Microservices en Python, Go, Nuxt, PHP ; IA JusMundi et recherche optimisée pour le contenu juridique international
		\item Continuité d'activité, exigences de sécurité élevées, accompagnement du déploiement IA (Engineering et Data Science)
		\end{itemize}}
	\\
	\twentyitem
		{2012 -}
		{2022}
		{Architecte Intégration Cloud}
		{\href{https://finastra.com}{Finastra} (ex-Misys / Thomson Reuters)}
		{Paris, France}
		{\begin{itemize}
		\item \textbf{Mission :} Transformation digitale et migration cloud — Trésorerie \& Marchés de capitaux (finastra.com)
		\item Mise en place de l'intégration continue et de l'automatisation
		\item Déploiement des pipelines Jenkins et Ansible ; réduction de 60\,\% des incidents d'intégration
		\item Sécurité : SAST/DAST, Klocwork, Fuzzing, détection de secrets, gestion des vulnérabilités
		\item Collaboration avec les équipes développement, QA et opérations pour les bonnes pratiques DevSecOps
		\end{itemize}}
	\\
	\twentyitem
		{2007 -}
		{2012}
		{Ingénieur Logiciel Senior}
		{\href{https://finastra.com}{Finastra}}
		{Paris, France}
		{\begin{itemize}
		\item \textbf{Mission :} Modules stock et comptabilité — Kondor+ Back-Office (K+TP)
		\item Architecture 3 tiers : services métier C++, Java J2EE, middleware CORBA
		\item Gestion des stocks en temps réel ; génération automatique des écritures bilancielles et hors bilan (conformité réglementaire)
		\item Optimisation des performances pour le traitement à fort volume
		\end{itemize}}
	\\
	\twentyitem
		{2005 -}
		{2010}
		{Ingénieur Logiciel}
		{Société Générale Corporate \& Investment Banking (via ESN)}
		{Paris, France}
		{\begin{itemize}
		\item \textbf{Mission :} Systèmes d'information back-office (EOLE, GATE)
		\item Développement et maintenance en C, PL/SQL, Java ; migration Oracle 8i vers 10g ; migration True64 vers Sun Solaris
		\item Support niveau 2, gestion des incidents de production, analyse des besoins avec les analystes métier
		\end{itemize}}
\end{twenty}


%----------------------------------------------------------------------------------------
%	 LANGUES
%----------------------------------------------------------------------------------------
\section{Langues}

\textbf{Français} — Langue maternelle | \textbf{Anglais} — Professionnel (TOEIC 750)

%----------------------------------------------------------------------------------------
%	 INFORMATIONS COMPLÉMENTAIRES
%----------------------------------------------------------------------------------------
\section{Informations complémentaires}

Disponibilité : ouvert aux missions en conseil et en contrat. Autorisation de travail : France et UE. Expérience du télétravail et des équipes distribuées. Mobilité envisageable (ex. Francfort).

%----------------------------------------------------------------------------------------
%	 RÉSUMÉ PROFESSIONNEL
%----------------------------------------------------------------------------------------


\end{document}
